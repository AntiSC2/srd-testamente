\documentclass{dsekkallelse}

\usepackage[T1]{fontenc}
\usepackage[utf8x]{inputenc}
\usepackage[swedish]{babel}

\setheader{Policy för Sektionsbilen}{Policydokument}{Antagen S17 2019-09-02}

\title{Policy för Sektionsbilen}
\author{Oskar Damkjaer}

\begin{document}

\section*{Policy för Sektionsbilen}

\subsection{Bakgrund}

Denna policy definerar reglerna som måste godkännas för att få köra D-sektionens bil, hädanefter kallad Drulle. Efterföljs inte reglerna kan förmånen att få låna Drulle dras in.

\subsection{Ändamål och användare}

För att få köra Drulle behöver du vara sektionsmedlem med körkort. Du får bara använda Drulle för sektionens verksamhet och bara köra inom Sverige.

\subsection{Bokning och nycklar}

Bokning görs via en styrelsemedlem och bokningslängden får ej överstiga 48 timmar. Undantag kan göras om styrelsen informerats och inte motsätter sig. Efter bokningen är godkänd kan du hämta ut nyckeln i styrelserummets kassaskåp eller från en styrelsemedlem. I nycklarna hänger ett bensinkort och ett axfoodkort. De ska hänga kvar där.

\subsection{Ansvar}

Du som bokat är ansvarig och är den enda som får köra. Du förblir ansvarig tills nyckeln är tillbakalämnad. Innan körning ser du över bilens skick och rapporterar skador till Källarmästaren, annars riskerar du att få ansvaret för skadorna. Övriga kostnader såsom parkering eller böter betalas av föraren. Om en skada uppkommer eller en olycka sker under körningen ska du omedelbart rapportera denna till Källarmästaren, oavsett storlek. Självrisken för sagda olycka betalas av sektionen givet att föraren inte varit oaktsam.

\subsection{Tankning}

Är tanken fylld till mindre än hälften ska du tanka och betala med bensinkortet i nyckelknippan. Använder du bensinkortet behöver du inte spara kvittot men du måste tanka på Circle K. Saknas bensinkortet, gör en utläggsräkning med D-Sektionen som mästeri.

\subsection{Parkering}
Drulle ska lämnas i ett välstädat skick, annars debiteras du för Drulles städning.
Parkering sker på parkeringsplatsen bakom E-huset (parkeringen där utedischot traditionellt hålls).

\newpage

\subsection{ Körjournal}

Journalen finns i Drulle och ska fyllas i vid varje körning. Är den full berätta detta för Källarmästaren. Du ska fylla i:

*   Dagens datum

*   Ditt utskott

*   Ditt namn

*   Ditt StiL-ID

*   Ditt ärende

*   Mätartalet vid start av din körning

*   Mätartalet vid avslut av din körning

*   Din signatur


\signature{För Styrelsen}{Oskar Damkjaer}{Studierådsordförande}


\end{document}
